\documentclass[]{article}
\usepackage{lmodern}
\usepackage{amssymb,amsmath}
\usepackage{ifxetex,ifluatex}
\usepackage{fixltx2e} % provides \textsubscript
\ifnum 0\ifxetex 1\fi\ifluatex 1\fi=0 % if pdftex
  \usepackage[T1]{fontenc}
  \usepackage[utf8]{inputenc}
\else % if luatex or xelatex
  \ifxetex
    \usepackage{mathspec}
  \else
    \usepackage{fontspec}
  \fi
  \defaultfontfeatures{Ligatures=TeX,Scale=MatchLowercase}
\fi
% use upquote if available, for straight quotes in verbatim environments
\IfFileExists{upquote.sty}{\usepackage{upquote}}{}
% use microtype if available
\IfFileExists{microtype.sty}{%
\usepackage{microtype}
\UseMicrotypeSet[protrusion]{basicmath} % disable protrusion for tt fonts
}{}
\usepackage[margin=1in]{geometry}
\usepackage{hyperref}
\hypersetup{unicode=true,
            pdftitle={Effect of the diet type on the C. elegans transcriptome},
            pdfauthor={Awan , Lotte, Nicky \& Melanie},
            pdfborder={0 0 0},
            breaklinks=true}
\urlstyle{same}  % don't use monospace font for urls
\usepackage{color}
\usepackage{fancyvrb}
\newcommand{\VerbBar}{|}
\newcommand{\VERB}{\Verb[commandchars=\\\{\}]}
\DefineVerbatimEnvironment{Highlighting}{Verbatim}{commandchars=\\\{\}}
% Add ',fontsize=\small' for more characters per line
\usepackage{framed}
\definecolor{shadecolor}{RGB}{248,248,248}
\newenvironment{Shaded}{\begin{snugshade}}{\end{snugshade}}
\newcommand{\KeywordTok}[1]{\textcolor[rgb]{0.13,0.29,0.53}{\textbf{#1}}}
\newcommand{\DataTypeTok}[1]{\textcolor[rgb]{0.13,0.29,0.53}{#1}}
\newcommand{\DecValTok}[1]{\textcolor[rgb]{0.00,0.00,0.81}{#1}}
\newcommand{\BaseNTok}[1]{\textcolor[rgb]{0.00,0.00,0.81}{#1}}
\newcommand{\FloatTok}[1]{\textcolor[rgb]{0.00,0.00,0.81}{#1}}
\newcommand{\ConstantTok}[1]{\textcolor[rgb]{0.00,0.00,0.00}{#1}}
\newcommand{\CharTok}[1]{\textcolor[rgb]{0.31,0.60,0.02}{#1}}
\newcommand{\SpecialCharTok}[1]{\textcolor[rgb]{0.00,0.00,0.00}{#1}}
\newcommand{\StringTok}[1]{\textcolor[rgb]{0.31,0.60,0.02}{#1}}
\newcommand{\VerbatimStringTok}[1]{\textcolor[rgb]{0.31,0.60,0.02}{#1}}
\newcommand{\SpecialStringTok}[1]{\textcolor[rgb]{0.31,0.60,0.02}{#1}}
\newcommand{\ImportTok}[1]{#1}
\newcommand{\CommentTok}[1]{\textcolor[rgb]{0.56,0.35,0.01}{\textit{#1}}}
\newcommand{\DocumentationTok}[1]{\textcolor[rgb]{0.56,0.35,0.01}{\textbf{\textit{#1}}}}
\newcommand{\AnnotationTok}[1]{\textcolor[rgb]{0.56,0.35,0.01}{\textbf{\textit{#1}}}}
\newcommand{\CommentVarTok}[1]{\textcolor[rgb]{0.56,0.35,0.01}{\textbf{\textit{#1}}}}
\newcommand{\OtherTok}[1]{\textcolor[rgb]{0.56,0.35,0.01}{#1}}
\newcommand{\FunctionTok}[1]{\textcolor[rgb]{0.00,0.00,0.00}{#1}}
\newcommand{\VariableTok}[1]{\textcolor[rgb]{0.00,0.00,0.00}{#1}}
\newcommand{\ControlFlowTok}[1]{\textcolor[rgb]{0.13,0.29,0.53}{\textbf{#1}}}
\newcommand{\OperatorTok}[1]{\textcolor[rgb]{0.81,0.36,0.00}{\textbf{#1}}}
\newcommand{\BuiltInTok}[1]{#1}
\newcommand{\ExtensionTok}[1]{#1}
\newcommand{\PreprocessorTok}[1]{\textcolor[rgb]{0.56,0.35,0.01}{\textit{#1}}}
\newcommand{\AttributeTok}[1]{\textcolor[rgb]{0.77,0.63,0.00}{#1}}
\newcommand{\RegionMarkerTok}[1]{#1}
\newcommand{\InformationTok}[1]{\textcolor[rgb]{0.56,0.35,0.01}{\textbf{\textit{#1}}}}
\newcommand{\WarningTok}[1]{\textcolor[rgb]{0.56,0.35,0.01}{\textbf{\textit{#1}}}}
\newcommand{\AlertTok}[1]{\textcolor[rgb]{0.94,0.16,0.16}{#1}}
\newcommand{\ErrorTok}[1]{\textcolor[rgb]{0.64,0.00,0.00}{\textbf{#1}}}
\newcommand{\NormalTok}[1]{#1}
\usepackage{graphicx,grffile}
\makeatletter
\def\maxwidth{\ifdim\Gin@nat@width>\linewidth\linewidth\else\Gin@nat@width\fi}
\def\maxheight{\ifdim\Gin@nat@height>\textheight\textheight\else\Gin@nat@height\fi}
\makeatother
% Scale images if necessary, so that they will not overflow the page
% margins by default, and it is still possible to overwrite the defaults
% using explicit options in \includegraphics[width, height, ...]{}
\setkeys{Gin}{width=\maxwidth,height=\maxheight,keepaspectratio}
\IfFileExists{parskip.sty}{%
\usepackage{parskip}
}{% else
\setlength{\parindent}{0pt}
\setlength{\parskip}{6pt plus 2pt minus 1pt}
}
\setlength{\emergencystretch}{3em}  % prevent overfull lines
\providecommand{\tightlist}{%
  \setlength{\itemsep}{0pt}\setlength{\parskip}{0pt}}
\setcounter{secnumdepth}{0}
% Redefines (sub)paragraphs to behave more like sections
\ifx\paragraph\undefined\else
\let\oldparagraph\paragraph
\renewcommand{\paragraph}[1]{\oldparagraph{#1}\mbox{}}
\fi
\ifx\subparagraph\undefined\else
\let\oldsubparagraph\subparagraph
\renewcommand{\subparagraph}[1]{\oldsubparagraph{#1}\mbox{}}
\fi

%%% Use protect on footnotes to avoid problems with footnotes in titles
\let\rmarkdownfootnote\footnote%
\def\footnote{\protect\rmarkdownfootnote}

%%% Change title format to be more compact
\usepackage{titling}

% Create subtitle command for use in maketitle
\providecommand{\subtitle}[1]{
  \posttitle{
    \begin{center}\large#1\end{center}
    }
}

\setlength{\droptitle}{-2em}

  \title{Effect of the diet type on the C. elegans transcriptome}
    \pretitle{\vspace{\droptitle}\centering\huge}
  \posttitle{\par}
    \author{Awan , Lotte, Nicky \& Melanie}
    \preauthor{\centering\large\emph}
  \postauthor{\par}
      \predate{\centering\large\emph}
  \postdate{\par}
    \date{13 juni 2019}


\begin{document}
\maketitle

\subsection{Introduction}\label{introduction}

The standard protocol for Caenorhabditis elegans growth and maintenance
is 20oC on an Escherichia coli diet.During this analysis we focused on
the changes in gene expression when the diet is changed to B.subtillis.
For each condition (diet type) we obtained three replicates.

\subsection{Imports}\label{imports}

Necessary libraries are loaded in first:

\begin{Shaded}
\begin{Highlighting}[]
\KeywordTok{library}\NormalTok{(openxlsx  , }\DataTypeTok{quietly =}\NormalTok{ T)}
\KeywordTok{library}\NormalTok{(edgeR     , }\DataTypeTok{quietly =}\NormalTok{ T)}
\KeywordTok{library}\NormalTok{(ggplot2   , }\DataTypeTok{quietly =}\NormalTok{ T)}
\KeywordTok{library}\NormalTok{(data.table, }\DataTypeTok{quietly =}\NormalTok{ T)}
\end{Highlighting}
\end{Shaded}

\subsection{Create countmatrix}\label{create-countmatrix}

Here we start with the count files for each sample. Each file contains
the geneIDs and their corresponding total count. All samples and their
count are put in one dataframe, counMatrix. Each column is named after
diet type and the last two digits of the SRR code.

\begin{Shaded}
\begin{Highlighting}[]
\CommentTok{#Open all count files}
\NormalTok{count85 <-}\StringTok{ }\NormalTok{(}\KeywordTok{read.csv}\NormalTok{(}\StringTok{"counts_SRR5832185.txt"}\NormalTok{, }\DataTypeTok{header =} \OtherTok{FALSE}\NormalTok{, }\DataTypeTok{sep =} \StringTok{"}\CharTok{\textbackslash{}t}\StringTok{"}\NormalTok{))}
\NormalTok{count86 <-}\StringTok{ }\NormalTok{(}\KeywordTok{read.csv}\NormalTok{(}\StringTok{"counts_SRR5832186.txt"}\NormalTok{, }\DataTypeTok{header =} \OtherTok{FALSE}\NormalTok{, }\DataTypeTok{sep =} \StringTok{"}\CharTok{\textbackslash{}t}\StringTok{"}\NormalTok{))}
\NormalTok{count87 <-}\StringTok{ }\NormalTok{(}\KeywordTok{read.csv}\NormalTok{(}\StringTok{"counts_SRR5832187.txt"}\NormalTok{, }\DataTypeTok{header =} \OtherTok{FALSE}\NormalTok{, }\DataTypeTok{sep =} \StringTok{"}\CharTok{\textbackslash{}t}\StringTok{"}\NormalTok{))}

\NormalTok{count94 <-}\StringTok{ }\NormalTok{(}\KeywordTok{read.csv}\NormalTok{(}\StringTok{"counts_SRR5832194.txt"}\NormalTok{, }\DataTypeTok{header =} \OtherTok{FALSE}\NormalTok{, }\DataTypeTok{sep =} \StringTok{"}\CharTok{\textbackslash{}t}\StringTok{"}\NormalTok{))}
\NormalTok{count95 <-}\StringTok{ }\NormalTok{(}\KeywordTok{read.csv}\NormalTok{(}\StringTok{"counts_SRR5832195.txt"}\NormalTok{, }\DataTypeTok{header =} \OtherTok{FALSE}\NormalTok{, }\DataTypeTok{sep =} \StringTok{"}\CharTok{\textbackslash{}t}\StringTok{"}\NormalTok{))}
\NormalTok{count96 <-}\StringTok{ }\NormalTok{(}\KeywordTok{read.csv}\NormalTok{(}\StringTok{"counts_SRR5832196.txt"}\NormalTok{, }\DataTypeTok{header =} \OtherTok{FALSE}\NormalTok{, }\DataTypeTok{sep =} \StringTok{"}\CharTok{\textbackslash{}t}\StringTok{"}\NormalTok{))}


\CommentTok{#split the chromosome and gene information}
\NormalTok{split1 <-}\StringTok{ }\KeywordTok{do.call}\NormalTok{(rbind ,}\KeywordTok{strsplit}\NormalTok{(}\KeywordTok{as.character}\NormalTok{(count85[,}\DecValTok{1}\NormalTok{]), }\StringTok{":"}\NormalTok{, }\DataTypeTok{fixed =} \OtherTok{TRUE}\NormalTok{))}
\NormalTok{split2 <-}\StringTok{ }\KeywordTok{do.call}\NormalTok{(rbind ,}\KeywordTok{strsplit}\NormalTok{(}\KeywordTok{as.character}\NormalTok{(split1[,}\DecValTok{2}\NormalTok{]) , }\StringTok{"."}\NormalTok{, }\DataTypeTok{fixed =} \OtherTok{TRUE}\NormalTok{))}

\CommentTok{#Add chromosome and gene information  to count85 dataframe and removes old column}
\NormalTok{count85 <-}\StringTok{ }\KeywordTok{cbind}\NormalTok{(split2, count85)}
\NormalTok{count85[,}\StringTok{"V1"}\NormalTok{] <-}\StringTok{ }\OtherTok{NULL}


\CommentTok{#creates count matrix }
\NormalTok{countMatrix <<-}\StringTok{ }\KeywordTok{do.call}\NormalTok{(cbind, }\KeywordTok{list}\NormalTok{(count85, count86}\OperatorTok{$}\NormalTok{V2, count87}\OperatorTok{$}\NormalTok{V2, count94}\OperatorTok{$}\NormalTok{V2, count95}\OperatorTok{$}\NormalTok{V2, count96}\OperatorTok{$}\NormalTok{V2))}
\KeywordTok{colnames}\NormalTok{(countMatrix) <-}\StringTok{ }\KeywordTok{c}\NormalTok{(}\StringTok{"Chromosome"}\NormalTok{, }\StringTok{"Gene"}\NormalTok{, }\StringTok{"E.coli_85"}\NormalTok{, }\StringTok{"E.coli_86"}\NormalTok{, }\StringTok{"E.coli_87"}\NormalTok{, }\StringTok{"B.subtillis_94"}\NormalTok{, }\StringTok{"B.subtillis_95"}\NormalTok{, }\StringTok{"B.subtillis_96"}\NormalTok{)}

\CommentTok{#change rownames to GeneID}
\KeywordTok{rownames}\NormalTok{(countMatrix) <-}\StringTok{ }\NormalTok{countMatrix[,}\StringTok{"Gene"}\NormalTok{]}

\KeywordTok{print}\NormalTok{(}\KeywordTok{head}\NormalTok{(countMatrix))}
\end{Highlighting}
\end{Shaded}

\begin{verbatim}
##         Chromosome  Gene E.coli_85 E.coli_86 E.coli_87 B.subtillis_94
## g6684 chrIII_pilon g6684         0         0         0              0
## g6685 chrIII_pilon g6685        10         6         6              7
## g6686 chrIII_pilon g6686         0         0         0              2
## g6687 chrIII_pilon g6687        26        14        14             20
## g6688 chrIII_pilon g6688         7         7         7              2
## g6689 chrIII_pilon g6689         1        24        24              0
##       B.subtillis_95 B.subtillis_96
## g6684              0              0
## g6685              6              7
## g6686              0              0
## g6687             26             27
## g6688              5             10
## g6689             58              5
\end{verbatim}

\subsection{Low count filtering}\label{low-count-filtering}

Genes with a low count are filtered out of the dataset (count
\textless{} 50).

\begin{Shaded}
\begin{Highlighting}[]
\CommentTok{# create DGE list}
\NormalTok{exp   <-}\StringTok{ }\KeywordTok{c}\NormalTok{(}\StringTok{"E.coli"}\NormalTok{, }\StringTok{"E.coli"}\NormalTok{, }\StringTok{"E.coli"}\NormalTok{, }\StringTok{"B.subtillis"}\NormalTok{, }\StringTok{"B.subtillis"}\NormalTok{, }\StringTok{"B.subtillis"}\NormalTok{)}
\NormalTok{group <-}\StringTok{ }\KeywordTok{factor}\NormalTok{(exp)}
\NormalTok{y     <-}\StringTok{ }\KeywordTok{DGEList}\NormalTok{(}\DataTypeTok{counts =}\NormalTok{ countMatrix[,}\DecValTok{3}\OperatorTok{:}\DecValTok{8}\NormalTok{], }\DataTypeTok{group =}\NormalTok{ group)}
  
\CommentTok{# select all genes with at least 50 counts per million (cpm) in two samples}
\NormalTok{keep.genes <-}\StringTok{ }\KeywordTok{rowSums}\NormalTok{(}\KeywordTok{cpm}\NormalTok{(y) }\OperatorTok{>}\StringTok{ }\DecValTok{50}\NormalTok{) }\OperatorTok{>=}\StringTok{ }\DecValTok{2}
\NormalTok{y <-}\StringTok{ }\NormalTok{y[keep.genes,]}
\end{Highlighting}
\end{Shaded}

\subsection{Normalization}\label{normalization}

Normalization of the data was done using the TMM method.

\begin{Shaded}
\begin{Highlighting}[]
\CommentTok{# recalculate the library size}
\NormalTok{y}\OperatorTok{$}\NormalTok{samples}\OperatorTok{$}\NormalTok{lib.size <-}\StringTok{ }\KeywordTok{colSums}\NormalTok{(y}\OperatorTok{$}\NormalTok{counts)}

\CommentTok{# normalization}
\NormalTok{y <-}\StringTok{ }\KeywordTok{calcNormFactors}\NormalTok{(y, }\DataTypeTok{method =} \StringTok{"TMM"}\NormalTok{)}
\end{Highlighting}
\end{Shaded}

\subsection{Create Design Matrix}\label{create-design-matrix}

Design matrix is made to organize the samples and their corresponding
conditions.

\begin{Shaded}
\begin{Highlighting}[]
\CommentTok{# create design matrix (samples grouped by conditions)}
\NormalTok{design <-}\StringTok{ }\KeywordTok{model.matrix}\NormalTok{(}\OperatorTok{~}\DecValTok{0}\OperatorTok{+}\NormalTok{group, }\DataTypeTok{data =}\NormalTok{ y}\OperatorTok{$}\NormalTok{samples)}
\KeywordTok{colnames}\NormalTok{(design) <-}\StringTok{ }\KeywordTok{levels}\NormalTok{(y}\OperatorTok{$}\NormalTok{samples}\OperatorTok{$}\NormalTok{group)}

\KeywordTok{print}\NormalTok{(}\KeywordTok{head}\NormalTok{(design))}
\end{Highlighting}
\end{Shaded}

\begin{verbatim}
##                B.subtillis E.coli
## E.coli_85                0      1
## E.coli_86                0      1
## E.coli_87                0      1
## B.subtillis_94           1      0
## B.subtillis_95           1      0
## B.subtillis_96           1      0
\end{verbatim}

\subsection{Estimate dispersion}\label{estimate-dispersion}

The common,Trended and Tagwise dispersion are calculated

\begin{Shaded}
\begin{Highlighting}[]
\CommentTok{# estimate dispersion}
\NormalTok{y <-}\StringTok{ }\KeywordTok{estimateGLMCommonDisp}\NormalTok{( y, design)}
\NormalTok{y <-}\StringTok{ }\KeywordTok{estimateGLMTrendedDisp}\NormalTok{(y, design, }\DataTypeTok{method =} \StringTok{"power"}\NormalTok{)}
\NormalTok{y <-}\StringTok{ }\KeywordTok{estimateGLMTagwiseDisp}\NormalTok{(y, design)}
\end{Highlighting}
\end{Shaded}

\subsection{Plot normalized data}\label{plot-normalized-data}

First we plot the multidimensional scaling, which is a multivariate data
analysis approach used for visualization of similarity and dissimilarity
in a two dimensional plot. Second, we plot the BCV, which is the square
root of the negative binomial dispersion.

\begin{Shaded}
\begin{Highlighting}[]
  \KeywordTok{plotMDS}\NormalTok{(y)}
\end{Highlighting}
\end{Shaded}

\includegraphics{markdown_script_files/figure-latex/unnamed-chunk-7-1.pdf}

\begin{Shaded}
\begin{Highlighting}[]
  \KeywordTok{plotBCV}\NormalTok{(y)}
\end{Highlighting}
\end{Shaded}

\includegraphics{markdown_script_files/figure-latex/unnamed-chunk-7-2.pdf}

\subsection{Determination of differentially expressed
genes}\label{determination-of-differentially-expressed-genes}

All significantly expressed genes (p-value \textless{} 0.05) are
filtered out and written to diff\_expressed\_genes.csv and data frame
res.

\begin{Shaded}
\begin{Highlighting}[]
\CommentTok{# Determine differentially expressed genes}
\NormalTok{fit <-}\StringTok{ }\KeywordTok{glmFit}\NormalTok{(y, design)}
\NormalTok{mc  <-}\StringTok{ }\KeywordTok{makeContrasts}\NormalTok{(}\DataTypeTok{exp.r=}\NormalTok{E.coli}\OperatorTok{-}\NormalTok{B.subtillis, }\DataTypeTok{levels =}\NormalTok{ design)}
\NormalTok{fit <-}\StringTok{ }\KeywordTok{glmLRT}\NormalTok{(fit, }\DataTypeTok{contrast =}\NormalTok{ mc)}
\NormalTok{res <-}\StringTok{ }\KeywordTok{topTags}\NormalTok{(fit, }\DataTypeTok{n =} \DecValTok{100000}\NormalTok{, }\DataTypeTok{p.value =} \FloatTok{0.05}\NormalTok{)}
\NormalTok{result_df <-}\StringTok{ }\KeywordTok{as.data.frame}\NormalTok{(}\KeywordTok{topTags}\NormalTok{(fit, }\DataTypeTok{n =} \DecValTok{100000}\NormalTok{, }\DataTypeTok{p.value =} \FloatTok{0.05}\NormalTok{))}
  
\KeywordTok{write.csv}\NormalTok{(res ,}\StringTok{"diff_expressed_genes.csv"}\NormalTok{, }\DataTypeTok{row.names =} \OtherTok{TRUE}\NormalTok{)}

\KeywordTok{print}\NormalTok{(}\KeywordTok{head}\NormalTok{(result_df))}
\end{Highlighting}
\end{Shaded}

\begin{verbatim}
##            logFC   logCPM       LR        PValue           FDR
## g12945 -5.954225 6.692446 883.6821 3.460733e-194 5.955922e-191
## g13180  2.399371 6.208840 310.9556  1.352216e-69  1.163582e-66
## g4550  -1.932292 6.472113 185.0521  3.823170e-42  2.193225e-39
## g16086 -3.079451 6.216210 164.8901  9.667564e-38  4.159469e-35
## g14478  2.380018 6.544507 156.2723  7.381659e-36  2.540767e-33
## g6024  -1.533203 6.380272 140.7693  1.807107e-32  5.183384e-30
\end{verbatim}

\subsection{Create input for future GSEA
analysis}\label{create-input-for-future-gsea-analysis}

For a gene set enrichment analysis ad txt file with genes and their
p-value and FDR are required.

\begin{Shaded}
\begin{Highlighting}[]
\NormalTok{  subSet <<-}\StringTok{ }\KeywordTok{as.data.frame}\NormalTok{(res[,}\DecValTok{4}\OperatorTok{:}\DecValTok{5}\NormalTok{])}
  \KeywordTok{setDT}\NormalTok{(subSet, }\DataTypeTok{keep.rownames =} \OtherTok{TRUE}\NormalTok{)[]}
\end{Highlighting}
\end{Shaded}

\begin{verbatim}
##          rn        PValue           FDR
##   1: g12945 3.460733e-194 5.955922e-191
##   2: g13180  1.352216e-69  1.163582e-66
##   3:  g4550  3.823170e-42  2.193225e-39
##   4: g16086  9.667564e-38  4.159469e-35
##   5: g14478  7.381659e-36  2.540767e-33
##  ---                                   
## 201:  g7925  5.582898e-03  4.780183e-02
## 202: g12821  5.624227e-03  4.791730e-02
## 203: g20364  5.707645e-03  4.838846e-02
## 204:  g1716  5.787202e-03  4.882242e-02
## 205:  g5218  5.897771e-03  4.951251e-02
\end{verbatim}

\begin{Shaded}
\begin{Highlighting}[]
  \KeywordTok{write.table}\NormalTok{(subSet, }\StringTok{"gsea_input.txt"}\NormalTok{, }\DataTypeTok{sep =} \StringTok{"}\CharTok{\textbackslash{}t}\StringTok{"}\NormalTok{, }\DataTypeTok{row.names =} \OtherTok{FALSE}\NormalTok{)}
\end{Highlighting}
\end{Shaded}


\end{document}
